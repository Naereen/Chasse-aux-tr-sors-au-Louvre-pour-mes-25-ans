\chapter{À propos}

\section*{Un petit mot des créateurs}

Nous espérons que ce livret et ce moment vont vous plaire !
On s'est bien amusé à le concevoir, c'est déjà ça...

Si un problème survient, ou que vous bloquez sur une énigme, que vous pensez avoir besoin d'aide\footnote{Par exemple si vous pensez qu'une erreur s'est glissée dans ce document…}, n'hésitez pas à nous contacter par téléphone ou nous appeler
\input{telephones.tex}.

Nous avons hâte de vous retrouver dans le jardin des tuileries afin de décerner le \emph{prix de la meilleure équipe !}
Toutes les équipes auront une récompense, ne vous inquiétez pas…


\section*{Aspect ``technique'' -- Remarques geek}
Ce document a été rédigé et compilé par mes soins (Lilian), en sélectionnant \emph{aléatoirement} \nbenigmes{} énigmes parmi une liste plus grande\footnote{\url{Goo.gl/JmPFeb}} de \totalnbenigmes{} énigmes.
%
Chaque énigme a été rédigée comme un petit document Markdown\footnote{\url{DaringFireball.net/projects/markdown/}},
qui est ensuite compilé en \LaTeX{} par \texttt{pandoc}\footnote{\url{pandoc.org/}}.
%
Le document principal est un simple document \LaTeX,
utilisant le style très épuré de Tufte-\LaTeX{}\footnote{\url{GitHub.com/Tufte-LaTeX/tufte-latex}}.
%
Les sources sont en accès libre\footnote{\url{GitHub.com/Naereen/Chasse-aux-tr-sors-au-Louvre-pour-mes-25-ans/}} et sous licence Creative Commons.

Ce recueil à été rédigé, imprimé et relié avec amour en janvier et février 2018.
Rendez-vous dans 25 ans pour la prochaine édition ?!

\hfill{} -- \emph{Lilian Besson \& Hélène Javelaud}.

\section*{Remarque importante}
Aucun bénéfice financier n'a été ni ne sera tiré de ce document.
C'est juste pour s'amuser !

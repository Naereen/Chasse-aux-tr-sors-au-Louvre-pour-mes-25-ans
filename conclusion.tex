\chapter{À propos}

\section*{Un petit mot des créateurs}

Nous espérons que ce livret et ce moment vont vous plaire !

Si vous bloquez sur une énigme ou que vous pensez avoir besoin d'aide\footnote{Par exemple si vous pensez qu'une erreur s'est glissée dans ce document…}, n'hésitez pas à nous contacter par téléphone ou nous appeler
\input{telephones.tex}.

Nous avons hâte de vous retrouver dans le jardin des tuileries afin de décerner le prix de la meilleure équipe !
Toutes les équipes auront une récompense, ne vous inquiétez pas…


\section*{Aspect ``technique''}
Ce document a été rédigé et compilé par mes soins, en sélectionnant \emph{aléatoirement} des énigmes parmi une liste plus grande\footnote{\url{https://github.com/Naereen/Chasse-aux-tr-sors-au-Louvre-pour-mes-25-ans/tree/master/src/}}.
Chaque livret contient \nbenigmes{} énigmes, tirées aléatoirement parmi \totalnbenigmes.

Chaque énigme a été rédigée et enregistrée comme un document Markdown\footnote{\url{https://daringfireball.net/projects/markdown/}},
qui est ensuite compilé en \LaTeX{} par \texttt{pandoc}\footnote{\url{http://pandoc.org/}}.

Le document principal est un simple document \LaTeX,
utilisant le style très épuré de Tufte-\LaTeX{}\footnote{\url{https://github.com/Tufte-LaTeX/tufte-latex}}.

Ce recueil à été rédigé et imprimé en janvier 2018.

-- Lilian Besson \& Hélène Javelaud.
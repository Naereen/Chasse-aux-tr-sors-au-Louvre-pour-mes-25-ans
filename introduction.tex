\chapter{Introduction}

\vspace*{-30pt}

Chères amies et chers amis, merci d'avoir répondu présent !
%
Vous voilà réunis par \textbf{équipe de \intervalparequipe{} personnes},
et vous avez \textbf{\nbenigmes{}\footnote{C'est pas nos $25$ ans pour rien !} tâches à effectuer}.
%
Vous avez \textbf{3 heures} : le musée évacue à \textbf{17h45} !
%
Rendez-vous\footnote{Et rendez-vous dans 25 ans pour la prochaine édition ?!} sous l'Arc de Triomphe du Carrousel, \textbf{à 18h.}
% la statue ``Retour de chasse''\footnote{Préparez-vous, les jeux de mots et clins d'œil sont nombreux !}
% Retour de chasse, Antonin Carlès (1888 ; 48° 51′ 49″ N, 2° 19′ 53″ E)
% https://fr.wikipedia.org/wiki/Liste_des_%C5%93uvres_publiques_du_1er_arrondissement_de_Paris#Jardins_des_Tuileries


\section*{Consignes}

L'ordre des énigmes est \emph{aléatoire}, elles ne sont triées ni par ordre chronologique, ni logique, ni spatial dans le musée, et n'ont aucune dépendance entres elles.
%
Toutes peuvent être résolues sans enfreindre le règlement intérieur du musée.
% il n'y a aucun piège.
\textbf{Pas de triche} : n'utilisez pas vos smartphones pour chercher sur Internet !

Munissez vous d'un appareil photo ou de vos \emph{smartphones} : \textbf{toutes les énigmes demandent de trouver une œuvre et de la prendre en photo}\footnote{Sans flash ! Pas de retouchage des photos, non plus !}. SVP, pensez bien à photographier aussi les étiquettes des œuvres !
Et si vous pouviez aussi noter le numéro de la salle de votre trouvaille, pour chaque énigme, directement sur le livret, ce serait chouette\footnote{On fera un joli petit site Internet qui montrera vos solutions pour chaque énigmes, avec leur localisation dans le musée.} !
Économisez votre batterie et relayez vous.
%
Vous êtes réunis dans une équipe, pensez donc à mobilisez les idées\footnote{Je n'ai pas pu m'empêcher de glisser quelques calculs mathématiques… -- \emph{Lilian}} de tout le monde (et faites connaissance) !

Tâchez d'être l'équipe la plus rapide ! La coopération entre les autres équipes n'est pas interdite…
Mais vous n'avez pas les mêmes énigmes qu'eux, et la meilleure équipe recevra un prix ce soir !
%
Un dernier conseil : restez calmes et discrets… il ne faut pas que les vigiles détectent que vous vous êtes lancés dans une chasse aux trésors…


\section*{Bonne chance !}
Affûtez votre regard, aiguisez votre attention, entrez dans le Musée du Louvre, et vous voilà près à affronter les autres équipes !

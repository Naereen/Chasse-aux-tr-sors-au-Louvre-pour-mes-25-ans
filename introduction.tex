\chapter{Introduction}

\vspace*{-30pt}

Chères amies et chers amis, merci d'avoir répondu présent !
%
Vous voilà réunis par \textbf{équipe de \nbparequipe{} personnes},
et vous avez \textbf{\nbenigmes{} tâches à effectuer}.
%
Vous avez \textbf{maximum $3$ heures} !
%
Rendez-vous devant la statue ``Retour de chasse'', dans le jardin des Tuileries, à partir de 17 heures.
% Retour de chasse, Antonin Carlès (1888 ; 48° 51′ 49″ N, 2° 19′ 53″ E)
% https://fr.wikipedia.org/wiki/Liste_des_%C5%93uvres_publiques_du_1er_arrondissement_de_Paris#Jardins_des_Tuileries


\section*{Règle du jeu}

Leur ordre est aléatoire, elles ne sont ni triées par ordre chronologique, ni logique, ni spatial dans le musée. Il n'y a pas de dépendances entre les énigmes.
%
Toutes les énigmes peuvent être résolues sans enfreindre le règlement intérieur du musée, il n'y a pas de piège.
Presque toutes les énigmes demandent de trouver une œuvre et de la prendre en photo, pensez bien à prendre aussi en photo les étiquettes des œuvres !
%
Vous êtes réunis dans une équipe que nous espérons aussi homogène que possible, donc pensez à mobilisez les compétences et connaissances de tout le monde !

Tâchez d'être les plus rapides ! La coopération entre les autres équipes n'est pas interdite…
Mais vous n'avez pas les mêmes énigmes qu'eux, et la meilleure équipe recevra un prix ce soir !
%
Un dernier conseil, restez calme et discret…, il ne faut pas que les vigiles détectent que vous vous êtes lancés dans une chasse aux trésors…


\section*{Bonne chance !}
Munissez vous d'un appareil photo ou de vos \emph{smartphones}, affûtez votre regard et aiguisez votre attention, vous voilà près à affronter les autres équipes !
